% Options for packages loaded elsewhere
\PassOptionsToPackage{unicode}{hyperref}
\PassOptionsToPackage{hyphens}{url}
\documentclass[
]{article}
\usepackage{xcolor}
\usepackage{amsmath,amssymb}
\setcounter{secnumdepth}{-\maxdimen} % remove section numbering
\usepackage{iftex}
\ifPDFTeX
  \usepackage[T1]{fontenc}
  \usepackage[utf8]{inputenc}
  \usepackage{textcomp} % provide euro and other symbols
\else % if luatex or xetex
  \usepackage{unicode-math} % this also loads fontspec
  \defaultfontfeatures{Scale=MatchLowercase}
  \defaultfontfeatures[\rmfamily]{Ligatures=TeX,Scale=1}
\fi
\usepackage{lmodern}
\ifPDFTeX\else
  % xetex/luatex font selection
\fi
% Use upquote if available, for straight quotes in verbatim environments
\IfFileExists{upquote.sty}{\usepackage{upquote}}{}
\IfFileExists{microtype.sty}{% use microtype if available
  \usepackage[]{microtype}
  \UseMicrotypeSet[protrusion]{basicmath} % disable protrusion for tt fonts
}{}
\makeatletter
\@ifundefined{KOMAClassName}{% if non-KOMA class
  \IfFileExists{parskip.sty}{%
    \usepackage{parskip}
  }{% else
    \setlength{\parindent}{0pt}
    \setlength{\parskip}{6pt plus 2pt minus 1pt}}
}{% if KOMA class
  \KOMAoptions{parskip=half}}
\makeatother
\usepackage{color}
\usepackage{fancyvrb}
\newcommand{\VerbBar}{|}
\newcommand{\VERB}{\Verb[commandchars=\\\{\}]}
\DefineVerbatimEnvironment{Highlighting}{Verbatim}{commandchars=\\\{\}}
% Add ',fontsize=\small' for more characters per line
\newenvironment{Shaded}{}{}
\newcommand{\AlertTok}[1]{\textcolor[rgb]{1.00,0.00,0.00}{\textbf{#1}}}
\newcommand{\AnnotationTok}[1]{\textcolor[rgb]{0.38,0.63,0.69}{\textbf{\textit{#1}}}}
\newcommand{\AttributeTok}[1]{\textcolor[rgb]{0.49,0.56,0.16}{#1}}
\newcommand{\BaseNTok}[1]{\textcolor[rgb]{0.25,0.63,0.44}{#1}}
\newcommand{\BuiltInTok}[1]{\textcolor[rgb]{0.00,0.50,0.00}{#1}}
\newcommand{\CharTok}[1]{\textcolor[rgb]{0.25,0.44,0.63}{#1}}
\newcommand{\CommentTok}[1]{\textcolor[rgb]{0.38,0.63,0.69}{\textit{#1}}}
\newcommand{\CommentVarTok}[1]{\textcolor[rgb]{0.38,0.63,0.69}{\textbf{\textit{#1}}}}
\newcommand{\ConstantTok}[1]{\textcolor[rgb]{0.53,0.00,0.00}{#1}}
\newcommand{\ControlFlowTok}[1]{\textcolor[rgb]{0.00,0.44,0.13}{\textbf{#1}}}
\newcommand{\DataTypeTok}[1]{\textcolor[rgb]{0.56,0.13,0.00}{#1}}
\newcommand{\DecValTok}[1]{\textcolor[rgb]{0.25,0.63,0.44}{#1}}
\newcommand{\DocumentationTok}[1]{\textcolor[rgb]{0.73,0.13,0.13}{\textit{#1}}}
\newcommand{\ErrorTok}[1]{\textcolor[rgb]{1.00,0.00,0.00}{\textbf{#1}}}
\newcommand{\ExtensionTok}[1]{#1}
\newcommand{\FloatTok}[1]{\textcolor[rgb]{0.25,0.63,0.44}{#1}}
\newcommand{\FunctionTok}[1]{\textcolor[rgb]{0.02,0.16,0.49}{#1}}
\newcommand{\ImportTok}[1]{\textcolor[rgb]{0.00,0.50,0.00}{\textbf{#1}}}
\newcommand{\InformationTok}[1]{\textcolor[rgb]{0.38,0.63,0.69}{\textbf{\textit{#1}}}}
\newcommand{\KeywordTok}[1]{\textcolor[rgb]{0.00,0.44,0.13}{\textbf{#1}}}
\newcommand{\NormalTok}[1]{#1}
\newcommand{\OperatorTok}[1]{\textcolor[rgb]{0.40,0.40,0.40}{#1}}
\newcommand{\OtherTok}[1]{\textcolor[rgb]{0.00,0.44,0.13}{#1}}
\newcommand{\PreprocessorTok}[1]{\textcolor[rgb]{0.74,0.48,0.00}{#1}}
\newcommand{\RegionMarkerTok}[1]{#1}
\newcommand{\SpecialCharTok}[1]{\textcolor[rgb]{0.25,0.44,0.63}{#1}}
\newcommand{\SpecialStringTok}[1]{\textcolor[rgb]{0.73,0.40,0.53}{#1}}
\newcommand{\StringTok}[1]{\textcolor[rgb]{0.25,0.44,0.63}{#1}}
\newcommand{\VariableTok}[1]{\textcolor[rgb]{0.10,0.09,0.49}{#1}}
\newcommand{\VerbatimStringTok}[1]{\textcolor[rgb]{0.25,0.44,0.63}{#1}}
\newcommand{\WarningTok}[1]{\textcolor[rgb]{0.38,0.63,0.69}{\textbf{\textit{#1}}}}
\setlength{\emergencystretch}{3em} % prevent overfull lines
\providecommand{\tightlist}{%
  \setlength{\itemsep}{0pt}\setlength{\parskip}{0pt}}
\usepackage[margin=1in]{geometry}
\usepackage{amsmath, amssymb}
\usepackage{booktabs}
\usepackage{tabularx}
\usepackage{bookmark}
\IfFileExists{xurl.sty}{\usepackage{xurl}}{} % add URL line breaks if available
\urlstyle{same}
\hypersetup{
  pdftitle={ABDUCTIO MVP},
  pdfauthor={David Joseph (adaptable)},
  hidelinks,
  pdfcreator={LaTeX via pandoc}}

\title{ABDUCTIO MVP}
\usepackage{etoolbox}
\makeatletter
\providecommand{\subtitle}[1]{% add subtitle to \maketitle
  \apptocmd{\@title}{\par {\large #1 \par}}{}{}
}
\makeatother
\subtitle{A Permutation-Invariant, Credit-Bounded Framework for
Symmetric Hypothesis Evaluation}
\author{David Joseph (adaptable)}
\date{December 19, 2025}

\begin{document}
\maketitle

\section{Abstract}\label{abstract}

ABDUCTIO MVP is a lightweight methodology for evaluating a mutually
exclusive, collectively exhaustive (MECE) set of hypotheses under strict
resource constraints. It is designed for domains in which one hypothesis
may appear "far-fetched" yet plausibly correct, and where common
reasoning failures arise from asymmetric scrutiny: evaluators decompose
the focal hypothesis into many requirements while leaving rivals vague
and atomic, allowing rivals to win by default.

ABDUCTIO MVP eliminates focal privilege and enforces \textbf{permutation
invariance}: the output assigned to any hypothesis is independent of
which hypothesis is chosen as a seed or listed first. The approach
combines (i) stand-alone, well-defined hypotheses (no complement
bundles), (ii) a fixed obligation template so each hypothesis must ``pay
the same kind of explanatory rent,'' (iii) a deterministic,
seed-invariant credit allocation policy, and (iv) an auditable ledger
update rule with an explicit open-world ``Other'' absorber.

The result is a publication-ready, implementation-ready framework that a
software engineer can implement directly without EVSI calculations,
Bayesian machinery, or complex statistical assumptions.

\section{1. Motivation and Problem}\label{motivation-and-problem}

Many controversial evaluations fail for a structural reason:

\begin{itemize}
\tightlist
\item
  Hypothesis \(H^\*\) (often ``far-fetched'') is decomposed into
  multiple subclaims.
\item
  Rival hypotheses \(R_i\) remain broad or underspecified.
\item
  Evidence undermining one subclaim of \(H^\*\) shifts weight to rivals.
\item
  Rivals gain weight not because they are supported, but because they
  were not required to articulate necessary commitments.
\end{itemize}

This is not merely a cognitive bias (``argument from incredulity'')---it
is also a \textbf{systems design} failure. If the procedure taxes some
hypotheses with specificity and not others, it bakes in unfairness.

ABDUCTIO MVP addresses this by requiring:

\begin{enumerate}
\tightlist
\item
  every hypothesis to be defined as a stand-alone mechanism, and
\item
  every hypothesis to be evaluated under the same obligation template
  and the same credit schedule, independent of ordering.
\end{enumerate}

\section{2. Core Requirement: Permutation
Invariance}\label{core-requirement-permutation-invariance}

\subsection{2.1 Informal statement}\label{informal-statement}

Given the same hypothesis set, the same evidence, and the same credit
budget, the final \((p,k)\) assigned to any hypothesis must not depend
on:

\begin{itemize}
\tightlist
\item
  which hypothesis was chosen as ``focal,''
\item
  the order hypotheses are listed,
\item
  the order evaluation steps are printed.
\end{itemize}

\subsection{2.2 Formal statement}\label{formal-statement}

Let \(H=\{h_1,\dots,h_n,h_{\text{other}}\}\) be a MECE set (named
hypotheses plus a catch-all Other). Let an engine \(F\) map: \[
F(H,E,B,\theta) \mapsto \{(p(h_i),k(h_i))\}_{i=1}^n \cup (p(h_{\text{other}}),k(h_{\text{other}}))
\] where \(E\) is evidence, \(B\) a credit budget, and \(\theta\)
configuration parameters.

Permutation invariance requires that for any permutation \(\pi\) of the
\textbf{named} hypotheses, \[
F(H,E,B,\theta) = F(\pi(H),E,B,\theta)
\] up to the same renaming/reordering of outputs.

\subsection{2.3 Design implications}\label{design-implications}

Permutation invariance forces three conditions:

\begin{enumerate}
\tightlist
\item
  \textbf{Semantic independence}: each hypothesis must be meaningful
  without reference to a ``seed.''
\item
  \textbf{Procedural symmetry}: credit allocation and stopping rules
  must not privilege any hypothesis.
\item
  \textbf{Determinism}: tie-breaking must not depend on presentation
  order.
\end{enumerate}

ABDUCTIO MVP implements all three.

\section{3. Design Principles}\label{design-principles}

\subsection{P1. Stand-alone
hypotheses}\label{p1.-stand-alone-hypotheses}

Each named hypothesis must be describable without mentioning any other
hypothesis. Prohibited: ``NOT H1,'' ``some mundane explanation,'' ``any
other cause,'' or umbrella OR-bundles as roots.

\subsection{P2. MECE + explicit Other}\label{p2.-mece-explicit-other}

The set of named hypotheses is intended to be mutually exclusive (ME)
and collectively exhaustive (CE). Collective exhaustiveness is
implemented pragmatically by always including:

\begin{itemize}
\tightlist
\item
  \(H_{\text{other}}\): ``Unknown/unmodeled explanation.''
\end{itemize}

\subsection{P3. No-free-probability}\label{p3.-no-free-probability}

Listing more subcases must not increase a hypothesis's probability.
Decomposition clarifies structure; it does not create credence.

\subsection{P4. Same burdens for all}\label{p4.-same-burdens-for-all}

Each hypothesis is evaluated through a fixed \textbf{obligation
template} (Section 6). This prevents one hypothesis from being saddled
with ``cosmic feasibility'' while rivals face only local plausibility
checks.

\subsection{P5. Credit-bounded
termination}\label{p5.-credit-bounded-termination}

Only two operations exist (Evaluate, Decompose), each costing 1 credit.
The process halts by budget or by meeting confidence thresholds.

\subsection{P6. Fully auditable}\label{p6.-fully-auditable}

Every update must be reproducible from logged arithmetic and rubric
scoring. No ``implicit'' ledger shifts are allowed.

\section{4. Data Model}\label{data-model}

\subsection{4.1 Hypothesis roots and
nodes}\label{hypothesis-roots-and-nodes}

A hypothesis is represented as a root node with an obligation template
and optional internal decomposition trees.

\begin{Shaded}
\begin{Highlighting}[]
\ImportTok{from}\NormalTok{ dataclasses }\ImportTok{import}\NormalTok{ dataclass, field}
\ImportTok{from}\NormalTok{ typing }\ImportTok{import}\NormalTok{ Optional, Literal, Dict, List, Tuple}

\NormalTok{Role }\OperatorTok{=}\NormalTok{ Literal[}\StringTok{"NEC"}\NormalTok{, }\StringTok{"EVID"}\NormalTok{]}
\NormalTok{DecompType }\OperatorTok{=}\NormalTok{ Literal[}\StringTok{"AND"}\NormalTok{, }\StringTok{"OR"}\NormalTok{]}
\NormalTok{OrMode }\OperatorTok{=}\NormalTok{ Literal[}\StringTok{"EXCLUSIVE"}\NormalTok{, }\StringTok{"INCLUSIVE"}\NormalTok{]}
\NormalTok{Scope }\OperatorTok{=}\NormalTok{ Literal[}\StringTok{"LOCAL\_ONLY"}\NormalTok{, }\StringTok{"GLOBAL\_TO\_TEMPLATE\_SLOT"}\NormalTok{, }\StringTok{"GLOBAL\_TO\_ROOT"}\NormalTok{]}

\AttributeTok{@dataclass}
\KeywordTok{class}\NormalTok{ Node:}
    \BuiltInTok{id}\NormalTok{: }\BuiltInTok{str}
\NormalTok{    statement: }\BuiltInTok{str}

    \CommentTok{\# Local scores for this node (not necessarily ledger probability)}
\NormalTok{    p: }\BuiltInTok{float} \OperatorTok{=} \FloatTok{1.0}          \CommentTok{\# default neutral for NEC nodes (see §7)}
\NormalTok{    k: }\BuiltInTok{float} \OperatorTok{=} \FloatTok{0.15}

    \CommentTok{\# Audit}
\NormalTok{    k\_rubric: Optional[Dict[}\BuiltInTok{str}\NormalTok{, }\BuiltInTok{int}\NormalTok{]] }\OperatorTok{=} \VariableTok{None}  \CommentTok{\# \{"A":0..2,"B":0..2,"C":0..2,"D":0..2\}}
\NormalTok{    factors: List[}\BuiltInTok{str}\NormalTok{] }\OperatorTok{=}\NormalTok{ field(default\_factory}\OperatorTok{=}\BuiltInTok{list}\NormalTok{)}
\NormalTok{    mind\_change: Optional[}\BuiltInTok{str}\NormalTok{] }\OperatorTok{=} \VariableTok{None}
\NormalTok{    evidence\_refs: List[}\BuiltInTok{str}\NormalTok{] }\OperatorTok{=}\NormalTok{ field(default\_factory}\OperatorTok{=}\BuiltInTok{list}\NormalTok{)}

    \CommentTok{\# Decomposition}
\NormalTok{    role: Optional[Role] }\OperatorTok{=} \VariableTok{None}
\NormalTok{    children: Dict[}\BuiltInTok{str}\NormalTok{, }\StringTok{"Node"}\NormalTok{] }\OperatorTok{=}\NormalTok{ field(default\_factory}\OperatorTok{=}\BuiltInTok{dict}\NormalTok{)}
\NormalTok{    decomp\_type: Optional[DecompType] }\OperatorTok{=} \VariableTok{None}
\NormalTok{    or\_mode: Optional[OrMode] }\OperatorTok{=} \VariableTok{None}

    \CommentTok{\# AND coupling for NEC children (pragmatic dependence weight)}
\NormalTok{    coupling: Optional[}\BuiltInTok{float}\NormalTok{] }\OperatorTok{=} \VariableTok{None}  \CommentTok{\# one of \{0.20, 0.50, 0.80, 0.95\}}

    \CommentTok{\# Accounting}
\NormalTok{    credits\_spent: }\BuiltInTok{int} \OperatorTok{=} \DecValTok{0}
\NormalTok{    status: Optional[}\BuiltInTok{str}\NormalTok{] }\OperatorTok{=} \VariableTok{None}      \CommentTok{\# "SCOPED", "UNSCOPED"}

\AttributeTok{@dataclass}
\KeywordTok{class}\NormalTok{ RootHypothesis:}
    \BuiltInTok{id}\NormalTok{: }\BuiltInTok{str}
\NormalTok{    statement: }\BuiltInTok{str}
\NormalTok{    exclusion\_clause: }\BuiltInTok{str}  \CommentTok{\# one line: what makes this not any other root}

    \CommentTok{\# Ledger probability (MECE bookkeeping)}
\NormalTok{    p\_ledger: }\BuiltInTok{float}
\NormalTok{    k\_root: }\BuiltInTok{float} \OperatorTok{=} \FloatTok{0.15}

    \CommentTok{\# Obligation slots (fixed template; §6)}
\NormalTok{    obligations: Dict[}\BuiltInTok{str}\NormalTok{, Node] }\OperatorTok{=}\NormalTok{ field(default\_factory}\OperatorTok{=}\BuiltInTok{dict}\NormalTok{)}

    \CommentTok{\# Audit}
\NormalTok{    credits\_spent: }\BuiltInTok{int} \OperatorTok{=} \DecValTok{0}

\AttributeTok{@dataclass}
\KeywordTok{class}\NormalTok{ HypothesisSet:}
\NormalTok{    roots: Dict[}\BuiltInTok{str}\NormalTok{, RootHypothesis]  }\CommentTok{\# includes "H\_other"}
\end{Highlighting}
\end{Shaded}

\subsection{4.2 Ledger invariants}\label{ledger-invariants}

Let named roots be \(H_1..H_n\) and \(H_{\text{other}}\). Maintain:

\begin{itemize}
\tightlist
\item
  \(p_{\text{ledger}}(h) \in [0,1]\)
\item
  \(\sum_{i=1}^n p_{\text{ledger}}(H_i) + p_{\text{ledger}}(H_{\text{other}}) = 1\)
\end{itemize}

\section{5. Cost Model}\label{cost-model}

Only two operations exist.

\begin{itemize}
\tightlist
\item
  \texttt{DECOMPOSE(target)} : 1 credit
\item
  \texttt{EVALUATE(target)} : 1 credit
\end{itemize}

Everything else (aggregation, ledger enforcement, scheduling) is
``free'' but must be logged.

\section{6. Obligation Template (Permutation-Invariance
Backbone)}\label{obligation-template-permutation-invariance-backbone}

Every named root hypothesis must be evaluated through the same template
of obligation slots. This guarantees that each hypothesis faces
comparable explanatory burdens.

\subsection{6.1 Required slots (default
MVP)}\label{required-slots-default-mvp}

Each root \(H_i\) must define four slots:

\begin{enumerate}
\tightlist
\item
  \textbf{Feasibility (general)} {[}NEC{]}

  \begin{itemize}
  \tightlist
  \item
    The mechanism is possible in principle.
  \end{itemize}
\item
  \textbf{Availability (context)} {[}NEC{]}

  \begin{itemize}
  \tightlist
  \item
    The mechanism is present/available in the specific
    time/place/context.
  \end{itemize}
\item
  \textbf{Fit to key features} {[}NEC{]}

  \begin{itemize}
  \tightlist
  \item
    The mechanism explains the core reported observations better than at
    least one competitor.
  \end{itemize}
\item
  \textbf{Defeater resistance} {[}NEC{]}

  \begin{itemize}
  \tightlist
  \item
    The strongest competitor-specific defeater does not apply.
  \end{itemize}
\end{enumerate}

These are expressed as NEC nodes. Additional EVID nodes are allowed but
may not be used to inflate probability.

\subsection{6.2 Template customization}\label{template-customization}

Implementations may add slots, but must:

\begin{itemize}
\tightlist
\item
  apply the same slots to all named roots, and
\item
  keep total slots small (4--7 recommended).
\end{itemize}

\subsection{6.3 Why this matters}\label{why-this-matters}

Without a template, decomposition can be weaponized: one hypothesis can
be loaded with ``universal feasibility'' while rivals get only vague
local stories. Template parity removes this asymmetry.

\section{7. Semantics of p within trees
(``No-free-probability'')}\label{semantics-of-p-within-trees-no-free-probability}

ABDUCTIO MVP distinguishes \textbf{ledger probability} from
\textbf{internal node p}:

\begin{itemize}
\tightlist
\item
  \(p_{\text{ledger}}(H_i)\): MECE bookkeeping probability over roots.
\item
  \(p(\text{NEC node})\): a \textbf{requirement-satisfaction score} for
  the obligation slot, interpreted as: ``How likely is it that this
  necessary condition is satisfied, given current evidence and
  assumptions?''
\end{itemize}

\subsection{7.1 Neutral defaults}\label{neutral-defaults}

To prevent ``conjunction crushing by listing,'' unassessed NEC nodes are
neutral:

\begin{itemize}
\tightlist
\item
  NEC nodes initialize at \(p=1.0\) (neutral multiplier)
\item
  with low confidence \(k=0.15\)
\end{itemize}

EVID nodes may initialize at \(p=0.5\) (uninformative) and \(k=0.15\).

\subsection{7.2 Consequence}\label{consequence}

Decomposition cannot lower a hypothesis merely by adding structure. Only
evaluated requirements can reduce the multiplier.

\section{8. Confidence k: Rubric and
Mapping}\label{confidence-k-rubric-and-mapping}

Confidence \(k\) is the stability/robustness of a credence estimate
under reasonable re-checking.

\subsection{8.1 Rubric (0--2 each)}\label{rubric-02-each}

A: Evidence Traceability B: Cross-Validation C: Sensitivity to
Assumptions D: Adversarial Resilience

Total \(T=A+B+C+D\) maps to:

\begin{itemize}
\tightlist
\item
  0--1 → 0.15
\item
  2--3 → 0.35
\item
  4--5 → 0.55
\item
  6--7 → 0.75
\item
  8 → 0.90
\end{itemize}

Guardrail: if any check = 0, cap \(k \le 0.55\).

\subsection{8.2 Root confidence}\label{root-confidence}

Root confidence \(k_{\text{root}}\) is the minimum \(k\) over assessed
NEC slots (conservative), optionally capped if UNSCOPED (Section 10).

\section{9. Decomposition Rules}\label{decomposition-rules}

\subsection{9.1 Root scoping is
mandatory}\label{root-scoping-is-mandatory}

All named roots must be decomposed into the obligation template before
any root can be accepted as ``well-scrutinized.''

\subsection{9.2 Additional decomposition within slots
(optional)}\label{additional-decomposition-within-slots-optional}

Each slot node may be decomposed further (2--5 children) when its
confidence is below threshold and credits remain.

\subsection{9.3 Coupling for AND nodes (within a
slot)}\label{coupling-for-and-nodes-within-a-slot}

When decomposing a slot into an AND of NEC children, choose coupling
\(c \in \{0.20,0.50,0.80,0.95\}\). Interpretation: pragmatic weight
toward bottlenecking (min) vs independence (product), not a statistical
coefficient.

Soft-AND for assessed children: \[
m = c \cdot p_{\min} + (1-c)\cdot p_{\prod}
\] where \(p_{\min}\) and \(p_{\prod}\) are computed over assessed NEC
children (unassessed treated as 1.0).

\section{10. Anti-Vagueness (UNSCOPED
rule)}\label{anti-vagueness-unscoped-rule}

A mechanism-like hypothesis must be able to state concrete necessary
commitments.

\subsection{Rule (root level)}\label{rule-root-level}

If a named root cannot instantiate the obligation template with
meaningful NEC statements, it is marked UNSCOPED and:

\begin{itemize}
\tightlist
\item
  cap \(k_{\text{root}} \le 0.40\),
\item
  it remains in the evaluation schedule until it becomes SCOPED or
  credits exhaust.
\end{itemize}

\subsection{Rule (slot level)}\label{rule-slot-level}

If a slot cannot be decomposed into at least 1 meaningful NEC statement,
cap that slot's \(k \le 0.40\).

This prevents ``winning by labels.''

\section{11. Aggregation: From obligations to root
proposal}\label{aggregation-from-obligations-to-root-proposal}

Let a root \(H_i\) have base ledger probability
\(p_{\text{base}} = p_{\text{ledger}}(H_i)\) at the time it is scoped
(template instantiated).

For each NEC slot \(s\), let its current satisfaction score be
\(p_s \in [0,1]\). Compute a \textbf{root multiplier}: \[
m_i = \prod_{s \in \text{slots}} p_s
\] but crucially, because unassessed slots start at \(p_s=1.0\), this
multiplier only decreases when a slot is actually evaluated and found
wanting.

Then propose a new root probability: \[
p_{\text{prop}}(H_i) = \mathrm{clip}(p_{\text{base}} \cdot m_i, 0, 1)
\]

\subsection{Notes}\label{notes}

\begin{itemize}
\tightlist
\item
  This is intentionally conservative: penalties arise from discovered
  weaknesses, not from the mere existence of multiple requirements.
\item
  Alternative within-slot AND aggregation can be used to compute each
  \(p_s\); the above treats the template as a product across slots
  (because these are distinct necessary obligations).
\end{itemize}

\section{12. Ledger Update with Other
Absorber}\label{ledger-update-with-other-absorber}

Ledger updates must be stable and auditable.

\subsection{12.1 Damping}\label{damping}

After computing \(p_{\text{prop}}(H_i)\) for any root, update: \[
p_{\text{ledger}}'(H_i) = (1-\alpha)\,p_{\text{ledger}}(H_i) + \alpha\,p_{\text{prop}}(H_i)
\] with \(\alpha \in (0,1]\) (default 0.4).

\subsection{12.2 Other absorber
invariant}\label{other-absorber-invariant}

Let \(S=\sum_{i=1}^n p_{\text{ledger}}'(H_i)\) over named roots
excluding Other.

\begin{itemize}
\tightlist
\item
  If \(S \le 1\): set \(p_{\text{ledger}}(H_{\text{other}})=1-S\).
\item
  If \(S > 1\): renormalize named roots only:
  \(p_{\text{ledger}}(H_i) = p_{\text{ledger}}'(H_i)/S\), set
  \(p_{\text{ledger}}(H_{\text{other}})=0\).
\end{itemize}

Log \(S\) and which branch was taken each time.

\section{13. Scheduling: Deterministic, Seed-Invariant Credit
Allocation}\label{scheduling-deterministic-seed-invariant-credit-allocation}

Permutation invariance requires that, given identical inputs, the same
multiset of operations be performed regardless of input ordering.

ABDUCTIO MVP uses \textbf{cycle scheduling} over a \textbf{frontier}
defined purely from the ledger state.

\subsection{13.1 Frontier definition (no focal
injection)}\label{frontier-definition-no-focal-injection}

Let leader be \(H_L = \arg\max p_{\text{ledger}}(H)\) over named roots.
Frontier: \[
F = \{H_i : p_{\text{ledger}}(H_i) \ge p_{\text{ledger}}(H_L) - \varepsilon\}
\] with \(\varepsilon\) default 0.05.

No ``seed'' or user-focal term may be unioned into frontier. (The UI may
highlight a focal, but the engine must not change scheduling.)

\subsection{13.2 Round-robin credit
slices}\label{round-robin-credit-slices}

Within each cycle:

\begin{itemize}
\tightlist
\item
  iterate over hypotheses in frontier ordered by a canonical ID (e.g.,
  SHA256 of root.statement),
\item
  for each hypothesis, perform exactly one operation chosen by the
  deterministic rule below,
\item
  decrement credits, update ledger, continue.
\end{itemize}

This round-robin is permutation-invariant because ordering is canonical,
not input order.

\subsection{13.3 Deterministic operation choice per
hypothesis}\label{deterministic-operation-choice-per-hypothesis}

For a root \(H_i\) in frontier, choose:

\begin{enumerate}
\tightlist
\item
  If \(H_i\) is UNSCOPED: DECOMPOSE (attempt to scope template or slot).
\item
  Else if any required slot is uninstantiated: DECOMPOSE to create
  missing slot node(s).
\item
  Else pick the slot \(s\) with lowest \(k\) (tie-break canonically by
  node ID):

  \begin{itemize}
  \tightlist
  \item
    If slot can be decomposed and \(k < \tau\): DECOMPOSE(slot)
  \item
    Else: EVALUATE(slot) or EVALUATE(slot's most critical child)
  \end{itemize}
\end{enumerate}

This ensures each frontier hypothesis is advanced comparably.

\subsection{13.4 Tie-breaking (mandatory)}\label{tie-breaking-mandatory}

All ties are broken by canonical ID derived from the statement text
(hash), never by input ordering.

\section{14. Stopping Conditions}\label{stopping-conditions}

Stop when any holds:

A) credits exhausted. B) For all hypotheses in the current frontier:

\begin{itemize}
\tightlist
\item
  root is SCOPED,
\item
  all template NEC slots have \(k \ge \tau\) (or credit exhaustion
  prevents further improvement).
\end{itemize}

C) No legal next operation exists (e.g., maximum decomposition depth
reached and no evaluable nodes remain).

\section{15. Evaluator and Decomposer Interfaces (Implementation
Contracts)}\label{evaluator-and-decomposer-interfaces-implementation-contracts}

\subsection{15.1 Evaluator contract (human or
agent)}\label{evaluator-contract-human-or-agent}

\texttt{evaluate(node,\ evidence)} returns:

\begin{itemize}
\tightlist
\item
  p in {[}0,1{]} (requirement-satisfaction for NEC; support score for
  EVID)
\item
  rubric A--D scores and derived k
\item
  1--3 short factors
\item
  mind\textsubscript{change} sentence
\item
  explicit assumption list if evidence is weak, each tagged with
  fragility (low/med/high)
\item
  evidence\textsubscript{refs} used (may be empty)
\end{itemize}

Constraints:

\begin{itemize}
\tightlist
\item
  If no evidence\textsubscript{refs}: enforce conservative p movement
  (implementation default: \textbar Δp\textbar{} \textless= 0.05 from
  prior node.p)
\item
  Evaluator must not reference ``seed'' or ``focal'' status.
\end{itemize}

\subsection{15.2 Decomposer contract}\label{decomposer-contract}

\texttt{decompose(node)} returns:

\begin{itemize}
\tightlist
\item
  2--5 children nodes, each labeled NEC or EVID
\item
  decomp\textsubscript{type} AND/OR; for OR, mode EXCLUSIVE/INCLUSIVE
\item
  for AND with NEC, coupling bucket c
\end{itemize}

Constraints:

\begin{itemize}
\tightlist
\item
  For root hypotheses, decomposer must instantiate the obligation
  template.
\item
  If cannot, mark UNSCOPED.
\end{itemize}

\section{16. MVP Algorithm (Pseudocode)}\label{mvp-algorithm-pseudocode}

\begin{Shaded}
\begin{Highlighting}[]
\NormalTok{inputs:}
\NormalTok{  claim text}
\NormalTok{  (optional) rivals list}
\NormalTok{  credits B}
\NormalTok{  tau, epsilon, gamma, alpha}
\NormalTok{  canonical\_id = hash(statement)}

\NormalTok{initialize:}
\NormalTok{  build named roots H1..Hn (claim + 3–5 single{-}mechanism rivals)}
\NormalTok{  add H\_other}
\NormalTok{  set uniform priors: p\_i=(1{-}gamma)/n ; p\_other=gamma}
\NormalTok{  set all k=0.15}
\NormalTok{  set all status=UNSCOPED initially (until template instantiated)}

\NormalTok{for cycle = 1.. while credits \textgreater{} 0:}
\NormalTok{  leader = argmax named roots by p\_ledger (tie{-}break by canonical\_id)}
\NormalTok{  frontier F = \{Hi : p\_ledger(Hi) \textgreater{}= p\_ledger(leader){-}epsilon\}}

\NormalTok{  order frontier by canonical\_id}
\NormalTok{  for Hi in F:}
\NormalTok{    if credits == 0: break}

\NormalTok{    choose operation deterministically:}
\NormalTok{      if Hi is UNSCOPED or missing template slots:}
\NormalTok{         DECOMPOSE(Hi)  \# instantiate template if possible}
\NormalTok{      else:}
\NormalTok{         pick slot s with lowest k (tie{-}break by canonical\_id)}
\NormalTok{         if can\_decompose(s) and k(s) \textless{} tau:}
\NormalTok{             DECOMPOSE(s)}
\NormalTok{         else:}
\NormalTok{             EVALUATE(s)}

\NormalTok{    spend 1 credit}
\NormalTok{    update credits\_spent on Hi and node}
\NormalTok{    recompute slot p\_s and k\_s via aggregation if needed}
\NormalTok{    compute p\_prop(Hi) = p\_base(Hi) * Π\_s p\_s}
\NormalTok{    apply damping to ledger p\_ledger(Hi)}
\NormalTok{    enforce Other absorber invariant}
\NormalTok{    log every arithmetic step and invariant checks}

\NormalTok{stop when stopping conditions met}
\NormalTok{output full audit trace}
\end{Highlighting}
\end{Shaded}

\section{17. Why ABDUCTIO MVP is
Permutation-Invariant}\label{why-abductio-mvp-is-permutation-invariant}

ABDUCTIO MVP achieves permutation invariance by construction:

\begin{enumerate}
\tightlist
\item
  No focal injection: frontier depends only on ledger state.
\item
  Canonical ordering: iteration order is defined by hash(statement), not
  by input list order.
\item
  Round-robin slicing: each frontier hypothesis receives equal
  opportunity per cycle.
\item
  No-free-probability semantics: decomposition cannot change ledger p;
  OR cannot inflate.
\item
  Template parity: each hypothesis is evaluated under the same
  obligation slots.
\item
  Deterministic tie-breaks everywhere.
\end{enumerate}

Given identical inputs, the engine performs the same operations in the
same canonical order and produces identical outputs, independent of the
seed.

\section{18. Practical Defaults}\label{practical-defaults}

\begin{itemize}
\tightlist
\item
  tau = 0.70
\item
  epsilon = 0.05
\item
  gamma = 0.20
\item
  alpha = 0.40
\item
  max\textsubscript{children} = 5
\item
  coupling\textsubscript{default} = 0.80
\item
  conservative delta p when no evidence: \textbar Δp\textbar{}
  \textless= 0.05 per evaluation
\end{itemize}

\section{19. Limitations and
Extensions}\label{limitations-and-extensions}

\subsection{Limitations}\label{limitations}

\begin{itemize}
\tightlist
\item
  This is not full Bayesian inference; it is a structured, auditable
  scoring-and-budgeting framework.
\item
  Results depend on the evaluator's discipline and evidence quality.
\item
  MECE is approximated; \(H_{\text{other}}\) absorbs residual
  uncertainty.
\end{itemize}

\subsection{Extensions}\label{extensions}

\begin{itemize}
\tightlist
\item
  Multi-assessor panels with aggregation rules for p and k.
\item
  Evidence objects with explicit likelihood impacts.
\item
  Specialized decomposers per domain (medicine, security, historical
  events).
\end{itemize}

\section{Appendix A: Minimal ``Well-Defined Hypothesis''
Checklist}\label{appendix-a-minimal-well-defined-hypothesis-checklist}

A root hypothesis must include:

\begin{itemize}
\tightlist
\item
  A mechanism statement (stand-alone).
\item
  An exclusion clause distinguishing it from other roots.
\item
  A template instantiation with NEC slots: feasibility, availability,
  fit, defeater resistance.
\end{itemize}

If it cannot, mark UNSCOPED and cap k.

\section{Appendix B: Canonical ID}\label{appendix-b-canonical-id}

Canonical ordering uses: \[
\text{canonical\_id}(h)=\text{SHA256}(\text{normalized\_statement\_text})
\] Normalization removes extra whitespace and lowercases text.

This ensures permutation invariance even if input ordering changes.

\section{Appendix C: Coupling buckets (within-slot
AND)}\label{appendix-c-coupling-buckets-within-slot-and}

Choose c as max of:

\begin{itemize}
\tightlist
\item
  Evidence overlap
\item
  Mechanism overlap
\item
  Failure-mode overlap
\end{itemize}

Buckets: 0.20, 0.50, 0.80, 0.95 Default 0.80 if unsure.

\end{document}
