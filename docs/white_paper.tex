% Created 2025-12-19 Fri 09:58
% Intended LaTeX compiler: pdflatex
\documentclass[11pt]{article}
\usepackage[utf8]{inputenc}
\usepackage[T1]{fontenc}
\usepackage{graphicx}
\usepackage{longtable}
\usepackage{wrapfig}
\usepackage{rotating}
\usepackage[normalem]{ulem}
\usepackage{amsmath}
\usepackage{amssymb}
\usepackage{capt-of}
\usepackage{hyperref}
\usepackage[margin=1in]{geometry}
\usepackage{amsmath, amssymb}
\usepackage{graphicx}
\usepackage{tabularx}
\usepackage{booktabs}
\author{David Joseph (adaptable)}
\date{December 19, 2025}
\title{ABDUCTIO-Lite v3\\\medskip
\large Symmetric, Credit-Bounded, Independence-Aware Hypothesis Evaluation}
\hypersetup{
 pdfauthor={David Joseph (adaptable)},
 pdftitle={ABDUCTIO-Lite v3},
 pdfkeywords={},
 pdfsubject={},
 pdfcreator={Emacs 30.2 (Org mode 9.7.11)}, 
 pdflang={English}}
\begin{document}

\maketitle
\setcounter{tocdepth}{2}
\tableofcontents

\section{Abstract}
\label{sec:orgeaeef68}
ABDUCTIO-Lite v3 is a lightweight, symmetric framework for evaluating a mutually exclusive set of hypotheses under strict resource limits. It addresses a common failure mode in controversial or “far-fetched” claims: evaluators often decompose and scrutinize the focal hypothesis while leaving rival hypotheses vague and atomic, allowing rivals to win by default.

ABDUCTIO-Lite v3 fixes this by enforcing scrutiny parity across all leading hypotheses using a simple credit budget, and by adding two minimal but enforceable standards:

\begin{enumerate}
\item a reasoner-agnostic \textbf{confidence (k)} rubric that treats confidence as \textbf{stability of credence under reasonable re-checking}, and
\item a simple \textbf{overlap (dependence)} heuristic for AND-decompositions.
\end{enumerate}

The result is a system that remains simple (no EVSI math), terminates naturally (credits), avoids probability blow-ups (via an explicit Other hypothesis), preserves conjunction penalties when independence is plausible, and gives each hypothesis a fair opportunity to rise or fall based on its own required structure.
\section{1. Problem Statement}
\label{sec:org0f2202d}
When evaluating a claim against alternatives, a common pathology appears:

\begin{itemize}
\item The focal hypothesis is decomposed into multiple sub-claims.
\item Rival hypotheses remain broad and underspecified.
\item Evidence that weakens a focal sub-claim shifts probability mass to rivals.
\item Rivals gain mass not because they were supported, but because they were never required to “pay” the explanatory cost of specificity.
\end{itemize}

This produces the argument-from-incredulity trap: a far-fetched hypothesis can be prematurely rejected because rivals are treated as catch-alls that absorb probability mass without being held to comparable explanatory standards.
\section{2. Design Goals and Constraints}
\label{sec:org3c0a903}
ABDUCTIO-Lite v3 is built to satisfy:

\begin{itemize}
\item \textbf{Perfect symmetry:} No special treatment for focal vs rivals.
\item \textbf{Credit-bounded recursion:} Termination is guaranteed by a simple credit budget.
\item \textbf{No EVSI math:} No expected-value-of-information calculations.
\item \textbf{MECE hypothesis set:} Hypotheses are treated as mutually exclusive; an explicit catch-all \textbf{Other} absorbs uncertainty.
\item \textbf{Preserve conjunction penalty:} When a hypothesis truly requires multiple independent conditions, it should become less likely accordingly.
\item \textbf{Avoid brittleness:} Do not let fragile independence assumptions or renormalization artifacts dominate outcomes.
\item \textbf{Reasoner-agnostic:} Works for humans, AI agents, and hybrid panels using the same minimal interchange format.
\end{itemize}
\section{3. Key Concepts}
\label{sec:orged3e220}
\subsection{3.1 Hypotheses as First-Class Objects}
\label{sec:org53bb483}
Every hypothesis is a first-class object that can be decomposed, evaluated, and updated under identical rules.
\subsection{3.2 Two Quantities per Node}
\label{sec:orgec2ea1b}
\begin{itemize}
\item \textbf{Credence} \(p\): current probability-like support in \([0,1]\).
\item \textbf{Confidence} \(k\): stability/robustness of the credence estimate; roughly, how much \(p\) would move under reasonable re-checking.
\end{itemize}

In Lite v3, \(k\) is scored via a simple, auditable rubric (Appendix B). No complex math is required.
\subsection{3.3 Credits}
\label{sec:org8876466}
A \textbf{credit} is the unit of evaluation cost. Typical costs:

\begin{itemize}
\item Evaluate a node: 1 credit
\item Decompose a node: 1 credit
\end{itemize}

Credits are the only mechanism needed to prevent infinite regress.
\section{4. Data Model}
\label{sec:org1890377}
\subsection{4.1 Hypothesis Node}
\label{sec:orgf281984}
\begin{verbatim}
class Hypothesis:
    id: str
    statement: str

    # Scores
    p: float          # credence in [0, 1]
    k: float          # confidence in [0, 1]

    # Tree
    children: list["Hypothesis"] = []
    decomp_type: str | None = None  # "AND" | "OR"

    # Accounting
    credits_spent: int = 0

    # Lite v3 additions
    child_role: str | None = None   # "NEC" (necessary) or "EVID" (supporting)
    and_mode: str | None = None     # "PRODUCT" | "CONSERVATIVE" | "SOFT"

    # Dependence / overlap estimate (Appendix C)
    overlap: float | None = None    # one of {0.20, 0.50, 0.80, 0.95}; default 0.80 if unknown

    # Confidence audit fields (Appendix B)
    k_checks: dict[str, int] | None = None  # {"A":0..2,"B":0..2,"C":0..2,"D":0..2}
    k_rationale: str | None = None
\end{verbatim}
\subsection{4.2 Hypothesis Set (MECE + Other)}
\label{sec:orgcd18704}
\begin{verbatim}
class HypothesisSet:
    hypotheses: dict[str, Hypothesis]  # includes "H_other"
    p_ledger: dict[str, float]         # MECE distribution over hypotheses

# Invariant: sum(p_ledger.values()) == 1.0
# H_other is always present and absorbs unmodeled uncertainty.
\end{verbatim}
\section{5. Two Rules That Prevent the “Catch-All Rival” Failure Mode}
\label{sec:orga483cf8}
\subsection{Rule A: Dominant Rival Challenge}
\label{sec:org428a9d5}
If any hypothesis becomes a leader without comparable scrutiny, it must defend its position.

\textbf{Trigger:} If a hypothesis becomes the leader (highest \(p\)) and is meaningfully under-scrutinized relative to the contender frontier, schedule it next until one of the stop conditions is met.

\textbf{Intent:} Rivals cannot win by vagueness.
\subsection{Rule B: Scrutiny Parity for the Contender Frontier}
\label{sec:org0ae3355}
Define the contender frontier:

\[
\text{frontier} = \{h : p(h) \ge \max_h p(h) - \varepsilon\} \cup \{\text{any user-designated focal}\}.
\]

\textbf{Parity condition:} No hypothesis in the frontier may lag far behind in credits spent.

Example:

\begin{itemize}
\item Let \(M\) be the median \(\text{credits\_spent}[h]\) over \(h\in\) frontier.
\item If any \(h\) has \(\text{credits\_spent}[h] < \rho\,M\) (e.g. \(\rho=0.8\)), schedule \(h\) next.
\end{itemize}

\textbf{Intent:} Ensure fair scrutiny among the hypotheses that actually matter.
\section{6. Decomposition Standard}
\label{sec:orgd19348c}
A decomposer must output:

\begin{itemize}
\item a small set of children (2–7; hard capped),
\item for each child, a role:
\begin{itemize}
\item \textbf{NEC} (necessary): if false, the parent becomes very unlikely,
\item \textbf{EVID} (evidential): supports/undermines, but is not strictly required,
\end{itemize}
\item decomposition type: AND or OR,
\item for AND with NEC children: an \textbf{and\textsubscript{mode}} and an \textbf{overlap} estimate (Appendix C).
\end{itemize}
\subsection{Guardrails}
\label{sec:org2f87ccd}
\begin{itemize}
\item \textbf{Arity cap:} max 7 children (default 5 recommended).
\item \textbf{Necessary test:} mark NEC only if the decomposer asserts: “If this child is false, the parent is substantially undermined.”
\item \textbf{Underspecified penalty:} if a mechanism-like hypothesis cannot produce at least 2 NEC children, it is marked underspecified (see §8.2).
\end{itemize}

These guardrails keep the system simple while preventing tree-shaping attacks.
\section{7. Aggregation Rules (Credence + Confidence)}
\label{sec:org9e23ca8}
Let:

\begin{itemize}
\item \(\text{NEC}\) = children with role == "NEC"
\item \(\text{EVID}\) = children with role == "EVID"
\end{itemize}
\subsection{7.1 OR Aggregation (NEC children)}
\label{sec:org3561c38}
OR means “any route suffices.”

\[
p_{\mathrm{OR}} = 1 - \prod_{i\in\mathrm{NEC}} (1 - p_i).
\]

Confidence for OR:

\[
k_{\mathrm{OR}} = \max_{i\in\mathrm{NEC}} k_i.
\]

If no NEC children exist, keep parent \(p,k\) as-is (or require decomposition correction).
\subsection{7.2 AND Aggregation: Three Modes}
\label{sec:orgae340d0}
AND means “all necessary conditions must hold.” Define:

\[
p_{\min} = \min_{i\in\mathrm{NEC}} p_i,\quad
p_{\prod} = \prod_{i\in\mathrm{NEC}} p_i.
\]

Let \(o\) be the overlap estimate in \(\{0.20,0.50,0.80,0.95\}\) (Appendix C). Default \(o=0.80\) if unknown.
\subsubsection{Mode 1: PRODUCT (full independence)}
\label{sec:orgd696879}
Use only when the decomposer explicitly certifies:

\begin{itemize}
\item NEC children are genuinely necessary, and
\item overlap is low (e.g. \(o\le 0.20\) or \(o\le 0.50\) with strong justification).
\end{itemize}

Then:

\[
p_{\mathrm{AND}} = p_{\prod}.
\]
\subsubsection{Mode 2: CONSERVATIVE (high dependence / unknown)}
\label{sec:orga8fe9c0}
Use when overlap is high/unknown, evidence overlaps, or decomposition is interpretive.

\[
p_{\mathrm{AND}} = p_{\min}.
\]
\subsubsection{Mode 3: SOFT (recommended default)}
\label{sec:org561446d}
A simple blend that retains conjunction intuition without catastrophic brittleness:

\[
p_{\mathrm{AND}} = o\,p_{\min} + (1-o)\,p_{\prod}.
\]

With default \(o=0.80\), AND behaves mostly like bottlenecking but still penalizes multiple weak requirements.
\subsection{7.3 EVID Handling (support without multiplication)}
\label{sec:org4ea36df}
EVID children should not be multiplied into \(p\) by default. They primarily affect confidence, and optionally provide a small \(p\) nudge.

Confidence update (simple conservative rule):

$\backslash$[
k\textsubscript{\text{NEC}} =
\begin{cases}
\min_{i\in\mathrm{NEC}} k_i & \text{if NEC nonempty}\\
k_{\text{parent prior}} & \text{otherwise}
\end{cases}
$\backslash$]

$\backslash$[
k\textsubscript{\text{EVID}} =
\begin{cases}
\max_{i\in\mathrm{EVID}} k_i & \text{if EVID nonempty}\\
0 & \text{otherwise}
\end{cases}
$\backslash$]

\[
k_{\text{parent}} = \min(k_{\text{NEC}},\; \max(k_{\text{EVID}},\; k_{\text{parent prior}})).
\]

Optional small \(p\) nudge (only if EVID exists):

\[
p_{\text{parent}} = 0.9\,p_{\text{NEC-agg}} + 0.1\,\mathrm{mean}_{i\in\mathrm{EVID}}(p_i).
\]

This keeps supporting reasons from masquerading as hard prerequisites.
\section{8. Confidence (k) Rules}
\label{sec:org8e71e37}
\subsection{8.1 Stopping Gate}
\label{sec:orgf0c0553}
Each hypothesis branch stops if:

\begin{itemize}
\item \(k_{\mathrm{eff}} \ge \tau\), or
\item credits for that branch are exhausted, or
\item decomposition is not possible.
\end{itemize}
\subsection{8.2 Underspecified Confidence Penalty}
\label{sec:org92f9485}
If \(k < \tau\) but the system cannot decompose the node into adequate structure, cap confidence:

\[
k \leftarrow \min(k, k_{\mathrm{cap}}),\quad k_{\mathrm{cap}}\approx 0.40.
\]

Interpretation: “I can’t specify what must be true for this to work” implies low robustness.
\subsection{8.3 Scrutiny-Aware Effective Confidence (optional)}
\label{sec:orgd6afa7f}
To prevent high-\(p\) hypotheses from coasting on low scrutiny, define:

\[
k_{\mathrm{eff}}(h) = k(h)\cdot \min\Bigl(1, \frac{\mathrm{credits\_spent}[h]}{\mathrm{parity\_target}}\Bigr),
\]

where \(\mathrm{parity\_target}\) is the median credits in the current frontier (or the leader’s credits).

This is optional but effective and simple.
\section{9. Ledger Update Without Probability Blow-Ups}
\label{sec:orgc44b8d7}
A major source of instability in naive systems is renormalizing tiny totals upward, causing a hypothesis to explode to near certainty.

ABDUCTIO-Lite v3 avoids this with an explicit Other absorber.
\subsection{Update Rule}
\label{sec:orgf2d8970}
After producing new proposed probabilities \(p_i\) for all non-Other hypotheses:

\begin{itemize}
\item Let \(S = \sum p_i\) over non-Other hypotheses.
\end{itemize}

If \(S \le 1\):

\begin{itemize}
\item keep all \(p_i\) as is
\item set \(p_{\mathrm{other}} = 1 - S\)
\end{itemize}

If \(S > 1\):

\begin{itemize}
\item renormalize only non-Other: \(p_i \leftarrow p_i/S\)
\item set \(p_{\mathrm{other}} = 0\)
\end{itemize}

This ensures missing mass represents uncertainty rather than forced certainty.
\section{10. Scheduling: The Core Loop (No EVSI)}
\label{sec:org7afd734}
ABDUCTIO-Lite v3 uses a simple prioritization policy:

\begin{enumerate}
\item Dominant Rival Challenge
\item Scrutiny Parity within the contender frontier
\item Otherwise, select the hypothesis with the highest unfinished business:
\end{enumerate}

\[
\mathrm{score}(h) = p(h)\,(1-k_{\mathrm{eff}}(h)).
\]
\subsection{Pseudocode}
\label{sec:org8df5666}
\begin{verbatim}
inputs:
  hypotheses MECE + H_other
  total_credits B
  confidence_threshold τ
  epsilon ε (frontier width)
  parity_ratio ρ (e.g. 0.8)

while B > 0:

  leader = argmax p_ledger
  frontier = {h : p_ledger[h] >= p_ledger[leader] - ε} ∪ {any user-designated focal}

  if exists h in frontier where credits_spent[h] < ρ * median_credits(frontier):
      target = argmin credits_spent in frontier  # Scrutiny Parity

  else:
      target = argmax over frontier of p_ledger[h] * (1 - k_eff[h])

  spend a small credit slice on target:
      evaluate(target)        # 1 credit
      if k_eff(target) < τ and can_decompose(target):
          decompose(target)   # 1 credit
          evaluate children with remaining slice

  aggregate target's local tree to update p/k
  update p_ledger using the H_other absorber rule
\end{verbatim}

Notes:

\begin{itemize}
\item If a leader is under-scrutinized, it will naturally be pulled by Scrutiny Parity.
\item In UIs that designate a “focal,” the frontier union with focal preserves user intent.
\end{itemize}
\section{11. Why Lite v3 Fixes the Common Failures}
\label{sec:org8353b95}
\subsection{11.1 Rivals cannot win by vagueness}
\label{sec:org363389c}
Leaders must pay scrutiny costs at parity. Broad catch-alls tend to weaken when forced to state NEC structure.
\subsection{11.2 Independence is preserved where real—and contained where not}
\label{sec:org3e1ffb7}
PRODUCT-AND remains available and powerful, but only when overlap is credibly low. Otherwise, CONSERVATIVE or SOFT prevents fake collapses.
\subsection{11.3 No renormalization explosions}
\label{sec:orgbf1cb57}
Uncertainty flows into \(H_{\mathrm{other}}\) instead of scaling tiny totals to 1.0.
\subsection{11.4 Decomposition cannot be weaponized}
\label{sec:orgb9142d5}
Arity caps and NEC/EVID roles prevent arbitrary multiplication of weak subclaims to crush a hypothesis.
\subsection{11.5 Confidence is reasoner-agnostic and auditable}
\label{sec:org701e8b3}
\(k\) is scored via a simple rubric and logged in a comparable way across humans and agents (Appendix B).
\section{12. Implementation Notes (MVP Scope)}
\label{sec:org69be7aa}
A minimal implementation can be \textasciitilde{}300–600 lines:

\begin{itemize}
\item Hypothesis node + set/ledger
\item Evaluator interface (human/LLM/manual)
\item Decomposer interface (human/LLM/manual)
\item Aggregator (AND/OR with modes)
\item Scheduler (frontier parity + unfinished business)
\item Audit trail: log every evaluation, decomposition, credits spent, and ledger update
\end{itemize}
\subsection{12.1 Interchange Card (Reasoner-Agnostic)}
\label{sec:org052240a}
This is the minimal schema humans and agents should produce.

\begin{verbatim}
{
  "proposition": "<string>",
  "assessment": {
    "p": 0.58,
    "k": 0.55,
    "k_checks": {"A": 1, "B": 1, "C": 1, "D": 1},
    "factors": [
      "short factor 1",
      "short factor 2",
      "short factor 3"
    ],
    "mind_change": "What would most change my mind (one sentence)."
  },
  "assessor": {
    "type": "human|ai|check",
    "role": "<string>",
    "id": "<string>"
  },
  "context": {
    "evidence_refs": ["<ref1>", "<ref2>"],
    "timestamp": "2025-12-19T12:00:00Z"
  }
}
\end{verbatim}
\subsection{12.2 Evaluator Contract}
\label{sec:org5068c54}
\begin{itemize}
\item \texttt{evaluate(hypothesis)} returns: \((p,k)\), 1–3 short factors, and one “mind-change” sentence.
\item For humans, Appendix B provides a one-minute worksheet.
\end{itemize}
\subsection{12.3 Decomposer Contract}
\label{sec:orgc6b9ae8}
\begin{itemize}
\item \texttt{decompose(hypothesis)} returns:
\begin{itemize}
\item children[] (2–7)
\item for each child: role NEC/EVID
\item decomp\textsubscript{type} AND/OR
\item for AND: and\textsubscript{mode} and overlap estimate via Appendix C
\end{itemize}
\end{itemize}
\section{13. Example Walkthrough Pattern (High Level)}
\label{sec:org6d5be1b}
Given hypotheses:

\begin{itemize}
\item \(H_1\): Far-fetched explanation
\item \(H_2\ldots H_n\): Conventional explanations
\item \(H_{\mathrm{other}}\): Unknown/other
\end{itemize}

Typical run behavior:

\begin{itemize}
\item Early: equal credit across hypotheses \(\rightarrow\) everyone gets at least one evaluation.
\item A conventional rival becomes leader by plausibility.
\item Scrutiny Parity forces the leader to decompose and pay explanatory costs.
\item If the rival is truly strong, it remains strong under structure; if it was a catch-all, it weakens.
\item As vague rivals weaken, mass shifts to other explicit hypotheses or into Other, not into artificial certainty.
\end{itemize}
\section{14. Conclusion}
\label{sec:org2beeb97}
ABDUCTIO-Lite v3 keeps the core appeal of a Lite system—simplicity, symmetry, and credit-bounded recursion—while fixing the practical problems that cause unfair rejections or brittle outcomes.

It ensures that every leading hypothesis earns its position through comparable scrutiny, preserves conjunction penalties when independence is warranted, avoids renormalization instability via an explicit Other absorber, and standardizes confidence as a reasoner-agnostic, auditable proxy for the stability of credence.
\section{Appendix A: Default Parameter Suggestions (Pragmatic)}
\label{sec:org43bf64f}
\begin{itemize}
\item \(\tau\) (confidence threshold): 0.70
\item \(\varepsilon\) (frontier width): 0.05
\item parity\textsubscript{ratio} \(\rho\): 0.80
\item max\textsubscript{children}: 5 (or 7)
\item overlap default \(o\): 0.80
\item and\textsubscript{mode} default: SOFT
\item \(k_{\mathrm{cap}}\) for underspecified: 0.40
\end{itemize}

These defaults bias toward robustness while preserving the ability for genuinely independent multi-requirement hypotheses to collapse appropriately.
\section{Appendix B: Confidence (k) as Stability of Credence (Human + AI Rubric)}
\label{sec:org2507acd}
\subsection{B.1 Plain-language definition}
\label{sec:orgdfd67f2}
Credence \(p\) is your current best guess. Confidence \(k\) is how stable that guess is under reasonable re-checking.

\begin{itemize}
\item High \(k\): \(p\) would barely move without genuinely new evidence.
\item Low \(k\): modest re-checks could move \(p\) a lot.
\end{itemize}
\subsection{B.2 The 4-Checks scorecard (0–2 each)}
\label{sec:org037881d}
Score each check as:

\begin{itemize}
\item 0 = Not yet
\item 1 = Partly
\item 2 = Yes, solidly
\end{itemize}
\subsubsection{Check A: Evidence Traceability}
\label{sec:orgb27da26}
Question: Can we point to specific evidence (or explicit premises) supporting this \(p\)?

\begin{itemize}
\item 0: mostly intuition / vague plausibility
\item 1: some concrete evidence, but thin/indirect
\item 2: clear, direct evidence (or crisp argument with explicit premises)
\end{itemize}
\subsubsection{Check B: Cross-Validation}
\label{sec:org8aa7b72}
Question: Would an independent re-check likely land near the same \(p\)?

\begin{itemize}
\item 0: one source/method
\item 1: some triangulation, but not independent
\item 2: strong triangulation across independent sources/methods
\end{itemize}
\subsubsection{Check C: Sensitivity to Assumptions}
\label{sec:org1fbf786}
Question: If reasonable assumptions change, does \(p\) swing?

\begin{itemize}
\item 0: very sensitive
\item 1: moderately sensitive
\item 2: robust
\end{itemize}
\subsubsection{Check D: Adversarial Resilience}
\label{sec:org5dd33cc}
Question: Have we faced the strongest plausible objections and rivals?

\begin{itemize}
\item 0: obvious defeaters not addressed
\item 1: some addressed, gaps remain
\item 2: main defeaters confronted
\end{itemize}
\subsection{B.3 Map total to \(k\)}
\label{sec:org268803b}
Let \(S=A+B+C+D\) (0–8):

\begin{itemize}
\item 0–1 \(\rightarrow\) \(k=0.15\)
\item 2–3 \(\rightarrow\) \(k=0.35\)
\item 4–5 \(\rightarrow\) \(k=0.55\)
\item 6–7 \(\rightarrow\) \(k=0.75\)
\item 8 \(\rightarrow\) \(k=0.90\)
\end{itemize}
\subsection{B.4 Recommended guardrails}
\label{sec:org948f614}
\begin{itemize}
\item Weakest-link cap: if any check is 0, cap \(k\le 0.55\).
\item Underspecified cap: if the hypothesis cannot state at least 2 NEC children when it should, cap \(k\le 0.40\).
\end{itemize}
\subsection{B.5 Logging requirements}
\label{sec:org6d0dfd5}
Each assessment records:

\begin{itemize}
\item \(p\), \(k\)
\item the four check scores A–D
\item 1 sentence: “What would most change my mind?”
\item 1–3 short factors behind \(p\)
\end{itemize}
\subsection{B.6 One-minute human worksheet}
\label{sec:org625169a}
\begin{quote}
Claim: \uline{\uline{\uline{\uline{\uline{\uline{\uline{\uline{\uline{\uline{\uline{\uline{\uline{\uline{\_\_}}}}}}}}}}}}}}
My current p: \uline{\_\_} /100 (or 0.00–1.00)

A Evidence Traceability (0/1/2): \uline{\_
B Cross-Validation (0/1/2): \_}
C Sensitivity (0/1/2): \uline{\_
D Adversarial Resilience (0/1/2): \_}
Total S: \uline{\_ → k: \_}

Top reasons (1–3 bullets):
\begin{itemize}
\item …
\item …
\item …
\end{itemize}

What would most change my mind (1 sentence):
\begin{itemize}
\item …
\end{itemize}
\end{quote}
\section{Appendix C: Overlap (Dependence) Estimation for AND Nodes (Human + AI)}
\label{sec:org8db89f4}
\subsection{C.1 Purpose (plain language)}
\label{sec:orgda3e0e0}
For AND with NEC children, overlap \(o\) estimates how much the NEC conditions “share the same fate.”

\begin{itemize}
\item Low overlap: conditions are mostly independent.
\item High overlap: conditions are coupled (shared evidence, mechanism, or failure mode).
\end{itemize}
\subsection{C.2 Bucket values (fixed for consistency)}
\label{sec:org98b5339}
Choose one of:$\backslash$
\(0.20\) (low), \(0.50\) (medium), \(0.80\) (high), \(0.95\) (very high / near-duplicate).
\subsection{C.3 The 3-layer method (take the maximum)}
\label{sec:org7c4aa1c}
Estimate:

\begin{itemize}
\item Evidence overlap \(E\)
\item Mechanism overlap \(M\)
\item Failure-mode overlap \(F\)
\end{itemize}

Then set:

\[
o = \max(E,M,F).
\]
\subsection{C.4 How to bucket each layer}
\label{sec:org0dc6516}
\subsubsection{Evidence overlap \(E\)}
\label{sec:org8b5cdab}
Ask: Do the NEC children rely on the same underlying observations?

\begin{itemize}
\item 0.95: essentially the same dataset/testimony/measurement
\item 0.80: heavily overlapping sources
\item 0.50: partial overlap
\item 0.20: distinct evidence streams
\end{itemize}
\subsubsection{Mechanism overlap \(M\)}
\label{sec:orga6a3e46}
Ask: Are the NEC children basically the same mechanism split into parts?

\begin{itemize}
\item 0.95: near-restatement / one implies the other
\item 0.80: hinge on one key mechanism assumption
\item 0.50: related mechanisms
\item 0.20: different mechanisms
\end{itemize}
\subsubsection{Failure-mode overlap \(F\)}
\label{sec:orgd72bc18}
Ask: Would one objection knock out multiple NEC children?

\begin{itemize}
\item 0.95: one debunk breaks most of them
\item 0.80: shared key vulnerability
\item 0.50: partial shared vulnerabilities
\item 0.20: different vulnerabilities
\end{itemize}
\subsection{C.5 Defaults and guardrails}
\label{sec:org716b283}
\begin{itemize}
\item Default: if unsure, set \(o=0.80\).
\item Merge rule: if \(o\ge 0.95\), usually merge the NEC children (or demote one to EVID).
\end{itemize}
\subsection{C.6 Quick human worksheet}
\label{sec:orgb01bab3}
\begin{quote}
Parent hypothesis: \uline{\uline{\uline{\uline{\uline{\uline{\uline{\uline{\uline{\uline{\uline{\uline{\_\_}}}}}}}}}}}}
NEC children: \uline{\uline{\uline{\uline{\uline{\uline{\uline{\uline{\uline{\uline{\uline{\uline{\uline{\uline{\uline{\_}}}}}}}}}}}}}}}

Evidence overlap E: 0.20 / 0.50 / 0.80 / 0.95
Mechanism overlap M: 0.20 / 0.50 / 0.80 / 0.95
Failure overlap F: 0.20 / 0.50 / 0.80 / 0.95

o = max(E,M,F): \uline{\_\_}
One-sentence justification:
\begin{itemize}
\item …
\end{itemize}
\end{quote}
\end{document}
